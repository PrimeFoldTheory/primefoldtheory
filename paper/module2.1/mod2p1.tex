\documentclass[12pt]{article}

% ---------- Core math & fonts ----------
\usepackage{amsmath,amssymb,amsfonts}
\usepackage{bm}               % bold math symbols
\usepackage{siunitx}          % units & aligned decimals in tables
\sisetup{detect-weight=true,detect-inline-weight=math}


% ---------- Graphics & floats ----------
\usepackage{graphicx}
\usepackage{float}
\usepackage{placeins}         % \FloatBarrier
\usepackage{booktabs}         % nice tables
\usepackage[labelfont=bf]{caption}
\usepackage{subcaption}       % (a)/(b) subfigures
\usepackage{microtype}        % better text spacing


% ---------- Handy symbols/macros ----------
\newcommand{\taubar}{\bar{\tau}}
\newcommand{\FoldTime}{\taubar}
\newcommand{\FoldDensity}{\Phi}
\newcommand{\Threshold}{\kappa}

% ---------- Figure/float defaults ----------
\captionsetup{font=small}
\setlength{\intextsep}{10pt}
\setlength{\textfloatsep}{12pt}
%--------------Python snippets------------------------
\usepackage{listings}
\lstset{
  basicstyle=\ttfamily\small,
  frame=single,
  breaklines=true,
  numbers=left,
  numberstyle=\tiny,
  xleftmargin=2em,
  language=Python
}

\title{Module 2.1: Cross-System Simulation of Fold-Time Invariance}
\author{Sean Sowden \\ \small Independent Researcher}
\date{August 31, 2025}


\begin{document}
\maketitle
\begin{center}
    \begingroup\small
\noindent © 2025 Sean Sowden. All Rights Reserved.
\par\endgroup

\end{center}

% Fix numbering so that Introduction starts at 2.1.1
\setcounter{section}{0}
\setcounter{secnumdepth}{2} % number sections & subsections only
\setcounter{tocdepth}{2}    % (optional) ToC up to subsections
\renewcommand{\thesection}{2.1.\arabic{section}}
\renewcommand{\thesubsection}{2.1.\arabic{section}.\arabic{subsection}}
\renewcommand{\thesubsubsection}{2.1.\arabic{section}.\arabic{subsection}.\arabic{subsubsection}}
%%%%%%%%%%%%%%%%%%%%%%%%%%%%%%%%%%%%%%%%%%%%%%%%%%%%%%%%%%%%%%%%%%%%%%
\section{Introduction}
In Module~1, the Fold law was defined through fold-time $\bar{\tau}$, fold density $\Phi$, and threshold $\kappa$. Module~2.1 extends this framework by testing whether fold-time invariance holds across distinct dynamical systems. We compare two minimal models: a driven lattice wave equation and a frictional block--spring ``quake toy.'' These models share no common medium, yet both accumulate structural resistance as fold-time until collapse at $\kappa$. If normalized fold-time cycles remain invariant across such disparate systems, this would support the claim that the Fold law is structural and universal, not medium-specific. To test this claim, we compare two toy systems under the same fold-time logic and assess whether normalized collapse cycles remain invariant.
%%%%%%%%%%%%%%%%%%%%%%%%%%%%%%%%%%%%%%%%%%%%%%%%%%%%%%%%%%%%%%%%%%%%%%%%%
\section{Field Models}
%%%%%%%%%%%%%%%%%%%%%%%%%%%%%%%%%%%%%%%%%%%%%%%%%%%%%%%%%%%%%
\subsubsection{SRF lattice field}
The first model is a simplified wave lattice, referred to here as the SRF (Structural Recursive Field). It evolves under a driven, damped wave equation on a discrete grid,
\begin{equation}
\frac{\partial^{2} u}{\partial t^{2}}
 - c^{2}\nabla^{2}u
 + \gamma \frac{\partial u}{\partial t}
 = F(t) ,
\label{eq:wave}
\end{equation}
where $u$ is the field variable, $c$ a wave speed, $\gamma$ a damping constant, and $F(t)$ the external drive.
The model is not intended to reproduce the physics of a specific medium, but rather to provide a minimal system in which gradients of $u$ represent structural resistance.
Fold-time $\bar{\tau}$ is defined as the time-integrated mean gradient magnitude,
\begin{equation}
\bar{\tau}(t) = \int \! \langle |\nabla u| \rangle \, dt ,
\label{eq:tau-srf}
\end{equation}
with collapse triggered when $\bar{\tau}\geq\kappa$, after which it is reset.
This abstraction allows the SRF lattice to act as a generic test bed for fold dynamics.
%%%%%%%%%%%%%%%%%%%%%%%%%%%%%%%%%%%%%%%%%%%%%%%%%%%%%%%%%%%%%
\subsubsection{Quake toy model}
The second model is a stick--slip block--spring system, commonly called a ``quake toy.'' This toy is widely used in seismology education, but here it is abstracted to a threshold-driven fold-time model.
It consists of a mass attached to a spring pulled at constant velocity across a frictional surface.
Its dynamics follow
\begin{equation}
m \ddot{x} = k \,\big(Vt - x\big) - F_{\mathrm{fric}}(\dot{x}) ,
\label{eq:quake}
\end{equation}
where $m$ is the block mass, $k$ the spring constant, $V$ the drive velocity, and $F_{\mathrm{fric}}$ the static/kinetic frictional force.
During stick phases, stress accumulates in the spring, and fold-time $\bar{\tau}$ is defined as the integrated stress,
\begin{equation}
\bar{\tau}(t) = \int \! \sigma(t)\, dt .
\label{eq:tau-quake}
\end{equation}
When the static friction threshold is exceeded, slip occurs, representing a collapse event, and $\bar{\tau}$ resets.
Although conceptually different from the SRF lattice, the quake toy follows the same fold-time logic: accumulation, threshold, collapse, reset.

\begin{figure}[htbp]
  \centering
  \begin{subfigure}{0.48\textwidth}
    \includegraphics[width=\linewidth]{mod2p1_srf_tau.png}
    \caption{SRF lattice}
    \label{fig:srf_tau}
  \end{subfigure}\hfill
  \begin{subfigure}{0.48\textwidth}
    \includegraphics[width=\linewidth]{mod2p1_quake_tau.png}
    \caption{Quake toy}
    \label{fig:quake_tau}
  \end{subfigure}
  \caption{Normalized fold-time traces $\bar{\tau}(t)$ for the SRF lattice and quake toy. Both accumulate until $\bar{\tau}\geq\kappa$, then reset.}
  \label{fig:tau_traces}
\end{figure}
\FloatBarrier
%%%%%%%%%%%%%%%%%%%%%%%%%%%%%%%%%%%%%%%%%%%%%%%%%%%%%%%%%%%%%%%%%
\section{Setup} Using the SRF lattice and stick–slip quake toy of Figure 1 above, we now test whether normalized fold-time cycles coincide across systems, dimensions, and parameters.

\section{Results}

\subsubsection{Direct comparison}
Both models produced fold-time traces $\bar{\tau}(t)$ cycling between accumulation and collapse.
When normalized to unit cycle length, the traces from the SRF lattice and quake toy aligned closely.
Across 18 valid runs, the median correlation coefficient was $r = 0.9994$ (IQR: 0.9989--0.9998), with RMSE below $2.1 \times 10^{-3}$.

\begin{figure}[htbp]
  \centering
  \includegraphics[width=0.60\textwidth]{mod2p1_trace_overlay.png}
  \caption{Normalized fold-time $\bar{\tau}$ for the SRF lattice (solid) and quake toy (dashed). Shaded band shows pointwise difference $\Delta$. Across 18 valid runs: median $r=0.9994$ (IQR 0.9989–0.9998), RMSE $< 2.1\times 10^{-3}$.}

  \label{fig:trace_overlay}
\end{figure}
\FloatBarrier
%%%%%%%%%%%%%%%%%%%%%%%%%%%%%%%%%%%%%%%%%%%%%%%%%%%%%%%%%%%%%%%%%
\subsubsection{Dimensional stability}
The SRF lattice was simulated in one, two, and three spatial dimensions.
In each case, fold-time $\bar{\tau}(t)$ cycled through accumulation and collapse as in the baseline 1-D run.
To test whether dimensionality altered the cycle profile, normalized fold-time traces from the first two full collapse cycles were extracted and averaged.
Overlaying the 1-D, 2-D, and 3-D traces showed near-perfect agreement (Fig.~\ref{fig:dim_overlay}). This confirms invariance persists a\begin{table}[htbp]
\centering
\caption{Stability metrics per dimension.
`events` = number of collapses detected;
`mean` = mean inter-collapse interval;
`std` = standard deviation;
`cv` = coefficient of variation.}
\label{tab:stability_metrics}
\begin{tabular}{lrrrr}
\hline
Dimension & Events & Mean Interval & Std Interval & CV \\
\hline
1D & 51  & 3.826  & 0.204 & 0.053 \\
2D & 53  & 2.894  & 0.215 & 0.074 \\
3D & 155 & 0.757  & 0.124 & 0.164 \\
\hline
\end{tabular}
\end{table}cross embedding dimension, not just in the simplest 1-D case.
Differences between curves were below one percent and consistent with numerical discretization noise.
This demonstrates that fold-time invariance is not a one-dimensional artifact but holds across higher-dimensional SRF lattices.

\begin{figure}[htbp]
  \centering
  \includegraphics[width=0.60\textwidth]{mod2p1_dimensional_overlay.png}
  \caption{Dimensional overlay: 1-D, 2-D, and 3-D SRF cycles overlap within numerical precision.}
  \label{fig:dim_overlay}
\end{figure}
\FloatBarrier


\subsubsection{Parameter sweeps}
Sweeps were performed across amplitude $A$, frequency $\omega$, and drive velocity $V$.
Runs were considered valid if at least three collapse events occurred in both models.
Out of 48 runs, 18 met this criterion.
Within these, median correlation remained $r \geq 0.999$, with little variance.
Invalid runs reflected insufficient drive rather than breakdown of invariance.

\begin{figure}[htbp]
  \centering
  \includegraphics[width=0.60\textwidth]{mod2p1_sweep_summary.png}
  \caption{Sweep summary across $(A,\omega,V)$ for valid runs ($\geq 3$ collapses in both models).
  Medians and interquartile ranges (IQR) are indicated by dashed lines.}
  \label{fig:sweep_summary}
\end{figure}
\FloatBarrier

\section{Discussion}

The results of Module 2.1 strengthen the case for fold-time invariance as a
structural principle. Three independent stress tests all point to the same conclusion:

\begin{enumerate}
  \item \textbf{Cross-system agreement.} The SRF lattice and stick--slip toy
  produce indistinguishable normalized fold-time cycles (Fig.~\ref{fig:trace_overlay}),
  with RMSE values on the order of $10^{-2}$ and correlations exceeding $0.999$.
  This indicates that invariance is not an artifact of a specific dynamical substrate.

  \item \textbf{Dimensional stability.} Extending the SRF lattice from one to
  three spatial dimensions produced no detectable change in the normalized
  accumulation--collapse profile (Fig.~\ref{fig:dim_overlay}).
  This shows that the invariance is not dependent on dimension, but on the fold-time
  dynamics themselves.

  \item \textbf{Robustness across parameters.} Parameter sweeps across amplitude,
  frequency, and velocity confirmed that invariance holds across a wide range of
  drive conditions (Fig.~\ref{fig:sweep_summary}). Median RMSE remained near
  $2.5 \times 10^{-2}$ with narrow interquartile range, while correlations clustered
  tightly at $r \approx 0.9994$. This statistical stability suggests the effect is
  not contingent on fine-tuning.
\end{enumerate}
%%%%%%%%%%%%%%%%%%%%%%%%%%%%%%%%%%%%%%%%%%%%%%%%%%%%%%%%%%%%%%%%%%%%%%%%%%%%%%%%%%
\section{Conclusion}
At its core, the fold-time dynamic reduces to a minimal conditional process:
accumulate below threshold, collapse at threshold, reset, and repeat.
This constitutes the simplest possible control structure—mathematically
equivalent to an IF–THEN–ELSE loop. What makes this notable is not the
complexity of the rule, but its persistence: the same conditional logic
reappears across distinct models (SRF lattice and stick–slip toy), across
spatial dimensions, and across parameter sweeps. The invariance of this
conditional cycle under stress testing suggests that fold-time embodies a
deep structural principle rather than a model-specific artifact.

It is important to note that the present models are simplified toy systems;
they are not intended as direct physical analogs. Rather, they serve to demonstrate
that invariance arises under minimal assumptions. Future work will extend these
tests to experimental data and more complex simulations, to probe the scope and
limits of fold-time invariance.
%%%%%%%%%%%%%%%%%%%%%%%%%%%%%%%%%%%%%%%%%%%%%%%%%%%%%%%%%%%%%%%%%%%%%%%%%%%%%%%%%%%%%%%
\appendix
%-------------------------------------------------------------
\section{Simulation parameters}
Parameter sweeps were conducted across amplitude $A \in [0.6, 1.4]$, frequency
$\omega \in [1.3, 2.1]$, and velocity $V \in [0.0015, 0.003]$. A run was considered
valid if both the SRF lattice and quake toy exhibited at least three collapse
events. Reported medians and interquartile ranges for RMSE and correlation are
calculated over the set of valid runs (see Fig.~\ref{fig:sweep_summary}).
%--------------------------------------------------------------------------------
\section{Simulation tuning and validity criteria}
\noindent\textbf{Parameter domains.} $A\in[0.6,1.4]$, $\omega\in[1.3,2.1]$, $V\in[0.0015,0.003]$.\\
\noindent\textbf{Run validity.} A run is \emph{valid} iff both models exhibit $\geq 3$ collapses.\\
\noindent\textbf{Metrics.} RMSE computed on $\bar{\tau}(u)$; Pearson $r$ computed on the same support.

%---------------python snippets------------------------------------
\section{Reproducibility: plotting snippets}

\subsection{Trace overlay (Fig.~\ref{fig:trace_overlay})}
\begin{lstlisting}
import pandas as pd, numpy as np, matplotlib.pyplot as plt
df = pd.read_csv("mod2p1_invariance_trace.csv")  # u, tau_norm_SRF, tau_norm_quake, delta
u,srf,quake = df.u.to_numpy(), df.tau_norm_SRF.to_numpy(), df.tau_norm_quake.to_numpy()
rmse = np.sqrt(np.mean((srf-quake)**2)); r = np.corrcoef(srf, quake)[0,1]
plt.figure(figsize=(6,4), dpi=200)
plt.plot(u, srf,  lw=2, label="SRF lattice")
plt.plot(u, quake,lw=2, ls="--", label="Quake toy")
plt.fill_between(u, srf, quake, alpha=0.25, label="Δ")
plt.xlabel("cycle phase (u)"); plt.ylabel(r"normalized $\bar{\tau}$")
plt.legend(frameon=False); plt.tight_layout()
plt.savefig("mod2p1_trace_overlay.png")
\end{lstlisting}

\subsection{Dimensional overlay (Fig.~\ref{fig:dim_overlay})}
\begin{lstlisting}
import pandas as pd, matplotlib.pyplot as plt
df = pd.read_csv("mod2p1_dimensional_overlay.csv")  # phase, 1D, 2D, 3D
plt.figure(figsize=(6,4), dpi=200)
for k in ["1D","2D","3D"]:
    plt.plot(df["phase"], df[k], lw=2, label=k)
plt.xlabel("cycle phase (u)"); plt.ylabel(r"normalized $\bar{\tau}$")
plt.legend(frameon=False); plt.tight_layout()
plt.savefig("mod2p1_dimensional_overlay.png")
\end{lstlisting}

\subsection{Sweep summary (Fig.~\ref{fig:sweep_summary})}
\begin{lstlisting}
import pandas as pd, matplotlib.pyplot as plt
df = pd.read_csv("mod2p1_sweep.csv")   # A, omega, V, RMSE, Corr, SRF_events, Quake_events, valid
df = df[df["valid"]==1]
plt.figure(figsize=(6,4), dpi=200)
plt.scatter(df["RMSE"], df["Corr"], alpha=0.7, s=25)
plt.axvline(df["RMSE"].median(), color="k", ls="--", lw=1)
plt.axhline(df["Corr"].median(), color="k", ls="--", lw=1)
plt.xlabel("RMSE"); plt.ylabel("Correlation (r)"); plt.tight_layout()
plt.savefig("mod2p1_sweep_summary.png")
\end{lstlisting}
%---------------------------------------------------
\section*{Acknowledgments}

The author designed and implemented all models and simulations, and is solely
responsible for the scientific content of this work. AI assistance (OpenAI’s GPT-5)
was used as a tool for text editing, LaTeX formatting, and organizational feedback.
All results, theoretical interpretations, and simulation designs are the author’s own.


\end{document}


\end{document}
